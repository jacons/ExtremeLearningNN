\documentclass[18p]{article}
\usepackage[T1]{fontenc}
\usepackage[english]{babel}
\usepackage{lastpage} %info sul # dell'ultima pagina del documento
\usepackage{geometry}
\usepackage{blindtext}
\usepackage[nottoc]{tocbibind} %Adds "References" to the table of contents
\usepackage{url}
\usepackage[colorlinks=true]{hyperref}
\usepackage{float}
\usepackage{subcaption}
\usepackage{color}
\usepackage[title]{appendix}
\usepackage{comment}
\usepackage[autostyle, italian=guillemets]{csquotes}
\usepackage[backend=biber, style=numeric]{biblatex}
\usepackage{guit}
\usepackage{algorithm}
\usepackage{algorithmic}
\addbibresource{Bibliografia.bib}
\usepackage{multicol}
\usepackage{minted}
\definecolor{bg}{rgb}{0.95,0.95,0.95}

\graphicspath{ {./images/} }
\usepackage{fancyhdr} %per mod


\usepackage{breakurl}%\url{...} può continare alla linea successiva. (si può andare a capo)     

\definecolor{Maroon}{cmyk}{0, 0.87, 0.68, 0.32}
\usepackage[colorlinks=true]{hyperref}
\hypersetup{
	colorlinks=true,
	citecolor=black,
	filecolor=black,
	linkcolor=black, % colore dei link interni
	urlcolor=Maroon  % colore dei link interniesterni
}

% Keywords command
\providecommand{\keywords}[1] {
  \small	
  \textbf{\textit{Keywords---}} #1
}

\usepackage{listings}

\usepackage{tikz}

\usepackage{longtable}
\usepackage{appendix} %aggiunge dei comandi per l'appendice
\usepackage{parskip} %aiuta LaTeX a trovare il miglior stile per i page break
\usepackage{setspace}

\usepackage{palatino}

\title{{\small Università di Pisa, Master Degree in Computer Science \\ Course of ''Computational Mathematics for Learning and Data Analysis'' (CM)} \\ \vspace{0.7em}
\huge{\textbf{Extreme learning model with Accelerated gradient and Cholesky factorization methods}} \\\vspace{0.9em} } 
\author{Iommi Andrea, Carfi Giacomo \\ \vspace{0.7em}
\vspace{0.4em}}

\date{\today}

\makeatletter

\def\@maketitle{
		\begin{center}
		{\@title }\\ 
		{\@author}\\ 
		\@date\\[4ex]
		\includegraphics[scale = 0.35]{logo-Pisa.jpeg}
\end{center}}
\makeatother

\begin{document}
    \maketitle

    \vspace{0.3cm}

    \section{Problem setup}
%    \section{Problem setup}
(M) is a so-called extreme learning, i.e., a neural network with one hidden layer, $y=W_2\sigma(W_1x)$ , where the weight matrix for the hidden layer $W_1$ is a fixed random matrix, $\sigma()$ is an elementwise activation function of your choice, and the output weight matrix $W_2$ is chosen by solving a linear least-squares problem with $L_2$ regularization.

(A1) is an algorithm of the class of accelerated gradient methods, applied to (M).

(A2) is a closed-form solution with the normal equations and your own implementation of Cholesky (or LDL) factorization.

\section{Introduction}

\subsection{Extreme Neural Network}
A variant of classical Feed-Forward Neural Network (FFNN) is the Extreme Neural Network (ENN). The difference is, typically, in the learning phase leaving the identical structure. In our project we take into account an ENN with one hidden layer. The peculiarity of this Network is that the learning procedure involves only the last layer (known as \textit{readout}), leaving the remaining layers untrained (i.e. with a fixed weight initialized randomly at the beginning). Several experiments show that ENN converges faster than traditional methods and the performance is comparable \cite{huang2004extreme}. ENN has been successfully applied in many real-time learning tasks for classification, clustering, and regression. The original paper can be found in \cite{huang2004extreme}. The key aspect of ENN is that the randomized hidden layer behaves as a random projection of the input. Often this projection increases the dimension of the input into a space where the data, hopefully, is more separable. It's important also the activation function (e.g Sigmoid, TanH, ReLU) between the random layer and the readout which ensure the non-linearity. The ENN performs the output in the following way, first it takes the input and perform the hidden state (Eq. \ref{first_layer}):

\begin{equation}
    \label{first_layer}
    \hat{h} = \sigma(W_1x)
\end{equation}

where $x = (x_1, x_2 ... x_{f-1},x_f)^T \in \mathbb{R}^f$ is the input and $f$ is the input dimension (a.k.a features). $W_1 \in \mathbb{R}^{h \times f}$ is a untrained and fixed weight. The result of (Eq. \ref{first_layer}) is $\hat{h} = (h_1,h_2...h_{h_1},h_{h_2}) \in \mathbb{R}^h$ hiddent state. Finally it carries out the output with the (Eq. \ref{second_layer}) where $W_2\in \mathbb{R}^{o \times h}$ and $o \in \mathbb{R}^{o}$:

\begin{equation}
    \label{second_layer}
    \mathbf{o} = W_2\hat{h}
\end{equation}

We can observe that in the last layer we do not deploy the activation function.


    \section{Related Works}
%    \input{1_Related_Works.tex}
    
    \section{Code and implementation}
    
    \section{Experimental set-up}
    \section{Experimental results}
    \section{Overall conclusions}
    
    \newpage
    \printbibliography
\end{document}